\section{Разработка генератора событий}
\subsection{План разработки и расчётов}
Для реализации генератора решено использовать объектно-ориентированный подход с целью обеспечить удобство дальнейших усовершенствований программы. В самой программе первоначально генерируется некоторое количество поперечных сечений струн в виде кругов фиксированного радиуса по заданному распределению (радиус струн соответствует параметрам распределения пропорционально реальному масштабу), затем производятся необходимые расчёты. Принципиальная схема работы программы представлена на рис. \ref{fig:DevSch} в приложении. Также в приложении (рис. \ref{fig:Strings}) представлена визуализация генерируемых струн. 

Для расчёта корреляций множественности и поперечного импульса адронов в переднем и заднем быстротных окнах (всюду далее величина корреляционного коэффицента будет обозначаться как $b$) необходимо сгенерировать соответственно значения множественности $n$ по сгенерированным струнами и значения поперечного имульса $p_t$ по струнной конфигурации и полученным значениям множественности. В качестве реализации учёта эффекта слияния струн $n$ и $p_t$ рассчитывается из плотности наложения (пересечения) струн друг на друга подсчётом количества струн в отдельно взятом кластере, а также подсчётом поперечной площади данного кластера (как наиболее физичное далее рассматриваться будет только глобальное слияние). 

Таким образом, разработка генератора разбивается на четыре этапа в порядке возрастания сложности, причём каждый следующий этап нуждается в завершённости предыдущих:
\begin{enumerate}[label=\arabic*.]
\item	Генерация струн.
\item	Поиск кластеров.
	\begin{enumerate}
	\item	Подсчёт количества струн в каждом кластере.
	\item	Вычисление площади каждого кластера.
	\end{enumerate}
\item	Генерация значений множественности и поперечного импульса.
\item	Расчёт FB корреляций множественности и поперечного импульса с учётом и без учёта слияния струн.
\end{enumerate}
%%%
\subsection{Выбор среды разработки}
Первоначально в качестве языка разработки был выбран MATLAB как очень удобный инструмент для быстрой и качественной реализации вычислительных алгоритмов. С помощью MATLAB были преодолены первые два этапа разработки, пока не был выявлен существенный недостаток данного решения -- слишком невысокая скорость расчётов по сравнению с более низкоуровневыми языками программирования, такими как C++ (приблизительно в 15-25 раз медленнее). На последнем впоследствии было заострено внимание, и далее было принято решение остановиться на нём для дальнейшей разработки. 

Скорость расчётов и оптимизация кода особенно важна в данной работе, так как предполагается расчёт большого количества симуляций. Так, в программе была задействована технология OpenMP, позволяющая создавать многопоточные приложения. 
%%%
\subsection{Структура программы}
Программа состоит из класса ``Simulation'', представляющего собой модель одной симуляции, и ``оболочки'', которая необходимое количество раз повторяет объект-симуляцию с загружаемым в неё извне средним количеством струн $\langle N \rangle$. Само значение $\langle N \rangle$ находится из значения параметра $\langle \eta \rangle$ (средняя плотность струн) через соотношение 
\begin{equation} \label{eq:eta}
	\langle \eta \rangle = \frac{\langle N \rangle \sigma_0}{S},
\end{equation}
где $\sigma_0$ -- площадь одной струны, $\sigma_0 = \pi r_s^2$, $r_s = 0.2 \div 0.3$ фм -- радиус струны; $S$ -- общая площадь взаимодействия в поперечном сечении.
Так, для энергий RHIC значение $\langle \eta \rangle \approx 3$, а для LHC $\langle \eta \rangle \approx 11$ \cite{RHICandLHC}. 

Класс симуляции состоит из нескольких основных (см. листинг \ref{lst:SimulationMethodsGeneral} в приложении) и вспомогательных (листинг \ref{lst:SimulationMethodsAdd} в приложении) функций-методов. Каждый из основных методов класса выполняется один раз за симуляцию в соответствии с порядком в листинге. Описание данных методов следует ниже по порядку.
%%%
\subsubsection{Генерация струн}
Чтобы сгенерировать струны, изначально следует задать их количество $N$. 
Для проверки корректности работы программы на этапах 1-3 достаточно полагать $N = \langle N \rangle$, а для расчёта $b$ (4 этап) решено генерировать $N$ с помощью распределения Пуассона с математическим ожиданием равным $\langle N \rangle$, что должно смоделировать реалистичный разброс $N$ от события к событию:
\begin{equation} \label{eq:poisson}
	f(N) = \frac{\langle N \rangle ^ N}{N!} e^{\langle N \rangle}.
\end{equation}

В качестве функций плотности распределения (PDF) координат струн в зависимости от того или иного расчёта берётся равномерное непрерывное распределение в круге, или же более физичное параболическое расрпеделение, приближающее гауссово:
\begin{equation} \label{eq:parab}
	z = \frac{2}{\pi} (-x^2 - y^2 + 1),
\end{equation}
где $2/\pi$ -- нормировочный множитель. 
Выбор распределения влияет также на определение площади взаимодействия: так, при равномерной PDF и максимальной центральности столкновения (в данной работе рассчитываются именно такие события) достаточно полагать $S = \pi R^2$ (площадь взаимодействия -- сечение ядра), где $R$ -- радиус ядра (7.5 фм), но при расчётах с параболическим распределением лучше брать $S$ как сумму площадей всех кластеров. 
Объяснение этому следующее: главным результатом является расчёт коэффициентов корреляции и сравнение их с теоретической асимптотой \cite{dissert} при $\langle \eta \rangle = 5 \div 11$; при данных плотностях, как правило, почти все струны уже слипаются в единый большой кластер, и, в то время как при равномерном распределении площадь этого кластера примерно равна площади ядра в поперечном сечении, при параболической PDF струны плотнее лежат к центру, оставляя некоторое пространство на периферии сечения ядра. 
Таким образом, сечение ядра не всегда можно считать площадью взаимодействия.

Неравномерная генерация координат струн происходит методом Неймана как универсальным методом генерации значений случайной величины по любой функции плотности распределения (в перспективе -- использование в качестве PDF довольно сложный потенциал Вудса-Саксона, поэтому использование метода Неймана оправдано). Метод Неймана заключается в следующем: на брусе, полностью покрывающем носитель функции плотности распределения $f(x)$, равномерно разыгрывается число $A$; затем равномерно разыгрывается число $B$ от 0 до 1, и если $B < f(A)$, то $A$ утверждается как величина, распределённая по $f(x)$.
%%%
\subsubsection{Заполнение графа и нахождение компонент связности; поиск максимальной компоненты}
Граф пересечений струн представляет из себя симметричную матрицу $N \times N$, заполненную номерами струн, которые пересекаются со струной с номером соответствующей строки или столбца, либо нулевых элементов (в случае, если струна под номером столбца или строки не пересекается ни с какой другой). Факт пересечения двух струн равносилен тому, что расстояние между их центрами меньше либо равно двум радиусам одной струны $2r_s$ (проверяется с помощью 5-й вспомогательной функции из листинга \ref{lst:SimulationMethodsAdd} в приложении). 

Для определения кластеров ищутся компоненты связности графа. Здесь используется метод Depth-first search: в качестве первой компоненты связности берётся первая струна $s_1$, затем производится поиск пересечённых с ней струн $\{s'_i\}$ и берётся первая пересечённая струна $s'_1$, далее ищутся струны $\{s''_i\}$, пересечённые с $s'_1$, и так далее, пока пересечённых струн не останется (до $s^{(n)}_1$); эта же процедура проводится для $s^{(n-1)}_2$ и всех остальных из $\{s^{(n-1)}_i\}$, из $\{s^{(n-2)}_i\}$ и так далее до самой $s'_2$. Одновременно ``проверенные'' струны вычёркиваются из списка поиска, и следующая компонента связности ищется по струне $s_j$, где $j > 1$ -- первый номер струны, не вошедшей в первую компоненту. Проще говоря, каждую компоненту связности можно представить как куст, состоящий из веток, веток этих веток и т.д. до самых листьев, а метод Depth-first search -- счётчик узлов веток (листья тоже считаются узлом -- на поверхности). Если этот счётчик установить в каком-нибудь узле внутри куста, то он будет на каждом шаге стремиться к листьям, чтобы посчитать сначала узлы на поверхности, затем уйдёт на первый слой узлов, предшествующий листьям, потом на второй и далее до самого глубокого слоя, где остановится. Слово ``Depth'' в названии метода произошло из терминологии в теории графов: глубокими уровнями компоненты называются именно поверхностные уровни куста. Наглядная схема начала обхода небольшого графа представлена на рис. \ref{fig:DFS} в приложении. Помимо неё в приложении представлен код, реализующий алгоритм поиска компонент методом DFS (листинг \ref{lst:DFS}).

%%%
\subsubsection{Количество струн в кластере и его площадь}
Количество струн $N_k$ в кластере находится как длина массива компоненты связности, соответствующей данному кластеру. Таким образом, например, можно найти кластер с максимальным количеством струн $N_{cl}$, которым будет являться самый длинный массив из всех компонент связности -- данный приём понадобится для некоторых тестов программы на первых этапах разработки. 

Менее тривиальной задачей является вычисление площади кластера $S_k$. Здесь на помощь приходит метод Монте-Карло вычисления площадей фигуры произвольной формы: фигурой в данном случае является граничный контур наложенных друг на друга кругов-струн. Суть метода в следующем: на прямоугольник, покрывающий фигуру, разбрасывается сетка из точек, и тогда площадь фигуры примерно равна площади прямоугольника, помноженной на отношение количества попавших на фигуру точек к количеству не попавших.
%%%
\subsubsection{Генерация множественности вперёд-назад}
С помощью описынных выше методов сперва вычисляется площадь каждого кластера $S_k$ и количество струн в них $N_k$. Следуя \cite{MulReduction}, средняя множественность $\langle n \rangle_k$ для $k$-го кластера (как было отмечено ранее, рассматривается случай глобального слияния) в данном быстротном интервале вычисляется как 
\begin{equation} \label{eq:nkaverage}
	\langle n \rangle_k = \mu_0 \frac{S_k}{\sigma_0}\sqrt{l_k}, \quad l_k = \frac{N_k \sigma_0}{S_k},
\end{equation}
где $\sigma_0$ -- площадь одной струны, $\mu_0$ -- средняя множественность от одной струны в данном быстротном интервале. Для единичного интервала быстроты $\mu_0 = 1.1$ в соответствии с \cite{Mu0} (в дальнейших расчётах бралось $\mu_0 = 1$). Формулу \ref{eq:nkaverage} можно упростить до 
\begin{equation} \label{eq:nksimple}
	\langle n \rangle_k = \mu_0 \sqrt{\frac{N_k S_k}{\sigma_0}},
\end{equation}
подставив явно выражение для $l_k$ в выражение для $\langle n \rangle_k$.

Найденное значение $\langle n \rangle_k$ показывает лишь среднее число частиц, рождённых кластером $k$. Пусть в симуляции $i$ кластер $k$ имеет среднее значение множественности $\langle n \rangle_{ki}$. Для генерации значения множественности в переднем $n^F_{ki}$ и заднем $n^B_{ki}$ быстротных окнах используется распределение Пуассона с математическим ожиданием равным $\langle n \rangle_{ki}$. Таким образом, для каждого кластера получается своя выборка значений $n^F_{ki}$ и $n^B_{ki}$, эти выборки суммируются для каждого события 
\begin{equation} \label{eq:nfnb}
\begin{split}
	n^F_i = \sum_{k = 1}^{M_i} n^F_{ki}, \\
	n^B_i = \sum_{k = 1}^{M_i} n^B_{ki},
\end{split}
\end{equation}
где $n^F_i$ и $n^B_i$ -- FB множественности в событии $i$, $M_i$ -- количество кластеров в событии $i$. Стоит отметить, что, так как речь идёт о каком-то отдельном событии, данные вычисления проводятся для фиксированного параметра $\eta$, сгенерированного для данного события. 

Представленный способ генерации множественности используется для ситуаций, где требуется учесть влияние слияния струн. Данные выкладки легко преобразуются для ситуаций, где требуется расчёт без учёта слияния: в этом случае $M_i = N_i$ (чисто струн в событии $i$), $N_k = 1$, $S_k = \sigma_0$, а значит, $\langle n \rangle_{ki} = \mu_0$.
%%%
\subsubsection{Генерация поперечного импульса вперёд-назад}
В соответствии с \cite{dissert} значения поперечного импульса генерируются не поочерёдно для каждого кластера, а сразу для всего события. 
Для этого необходимо знать вычисленные по описанной выше схеме значения $n^F_{ki}$ и $n^B_{ki}$. 
Тогда значения $(p_t)_i^F$, $(p_t)_i^B$ в соответствии с формулами 6.137-6.138 из \cite{dissert} (согласно центральной предельной теореме, распределение $(p_t)_i^F$ и $(p_t)_i^B$ является гауссовым, так как данные величины являются суммой поперечных импульсов частиц из кластеров -- то есть величин, распределённых одинаково) распределены по гауссу, то есть функция распределения выглядит следующим образом (с точностью до замены F на B):
\begin{equation} \label{eq:pfpb}
\begin{split}
	f((p_t)_i^F) = \frac{1}{\sqrt{2\pi} \sigma_{(p_t)_i^F}} \exp{\left( - \frac{((p_t)_i^F - \overline{(p_t)_i^F})^2}{2(\sigma_{(p_t)_i^F})^2} \right)}, \\
	\overline{(p_t)_i^F} = \frac{\overline{p}}{n_i^F} \sum_{k = 1}^{M_i} n_k \sqrt[4]{\eta_k} = \overline{p} \cdot p_\Sigma, \qquad \quad \quad \\
	\sigma_{(p_t)_i^F}^2 = \frac{\sigma_p^2}{(n_i^F)^2} \sum_{k = 1}^{M_i} n_k \sqrt{\eta_k} = \sigma_p^2 \cdot \sigma_\Sigma^2, \quad \quad \quad
\end{split}
\end{equation}
где $\overline{(p_t)_i^F}$ и $\sigma_{(p_t)_i^F}$ -- среднее и корень из дисперсии величины $(p_t)_i^F$; $\overline{p}$ и $\sigma_p^2$ -- некоторые константы, которые, согласно \cite{dissert}, связаны ссотношением $\sigma_p^2 = \gamma \overline{p}$; для расчётов бралось $\gamma = 1/\sqrt{2}$. 
Для расчётов корреляционного коэффициента $b$ можно не задавать явно $\overline{(p_t)_i^F}$ и $\sigma_{(p_t)_i^F}$, а достаточно преобразовать распределение \ref{eq:pfpb}, вынеся $\overline{(p_t)_i^F}$ и $\sigma_{(p_t)_i^F}$ из 
\begin{equation} \label{eq:pfpbBetter}
\begin{split}
	f \left( \frac{ (p_t)_i^F } { \overline{p} } \right) = \frac{1}{\sqrt{2 \pi} \sigma_p \sigma_\Sigma} \exp{ \left( - \frac{ \overline{p}^2 \left( \frac{ (p_t)_i^F }{ \overline{p} } - p_\Sigma \right)^2} { 2 \sigma_p^2 \sigma_\Sigma^2 } \right) } \Longrightarrow \\
	f \left( \frac{ (p_t)_i^F } { \overline{p} } \right) = \frac{1}{\sqrt{2 \pi} \sigma_p \sigma_\Sigma} \exp{ \left( - \frac{ \left( \frac{ (p_t)_i^F }{ \overline{p} } - p_\Sigma \right)^2} { 2 \gamma^2 \sigma_\Sigma^2 } \right) }, \quad \enspace \enspace
\end{split}
\end{equation}
и тогда можно задать только $\gamma$, так как при расчёте $b$, как будет описано далее, входит только частное статистически равных величин, распределённых по формуле \ref{eq:pfpbBetter}.
%%%
\subsubsection{Расчёт корреляционного коэффициента}
Выборки значений $\{ (n_F)_i \}$, $\{ (n_B)_i \}$, $\{ (p_F)_i \}$ и $\{ (p_B)_i \}$ непосредственно используются в расчёте $b$ с помощью вспомогательного метода 1 (см. листинг \ref{lst:SimulationMethodsAdd} в приложении). Для фиксированного $\langle \eta \rangle$ в соответствии с \cite{bStatement} величина $b$ находится как 
\begin{equation} \label{eq:bCalc}
\begin{split}
	b_{nn} = \frac{\langle n_F n_B \rangle - \langle n_F \rangle \langle n_B \rangle}{\langle n_F^2 \rangle - \langle n_F \rangle^2}, \\
	b_{p_tp_t} = \frac{\langle p_F p_B \rangle - \langle p_F \rangle \langle p_B \rangle}{\langle p_F^2 \rangle - \langle p_F \rangle^2},
\end{split}
\end{equation}
где усреднение происходит по всем симуляциям. Финальной проверкой корректности расчётов является сравнение значений $b$ с теоретическими асимптотами из \cite{dissert}:
\begin{equation} \label{eq:bTheor}
\begin{split}
	b_{nn} = \frac{1}{1 + 4 \cdot \sqrt{\langle \eta \rangle}}, \quad \\
	b_{p_tp_t} = \frac{1}{1 + 16 \cdot \gamma^2 \cdot \sqrt{\langle \eta \rangle}},
\end{split}
\end{equation}
где $\gamma^2$ -- такое же, как было описано выше, то есть $\gamma = 1/\sqrt{2}$.
