\documentclass[12pt,a4paper]{article}

%packages
\usepackage[utf8]{inputenc}
\usepackage[T1,T2A]{fontenc}
\usepackage[english,russian]{babel}
\usepackage{graphicx}
\usepackage{longtable}
%\usepackage{subcaption}
\usepackage{subfig}
\usepackage{float}
\usepackage{color, colortbl}
\usepackage{amsmath}
\usepackage{siunitx}
\usepackage{booktabs}

\setlength{\parindent}{0em}
\setlength{\parskip}{1em}

\definecolor{Gray}{gray}{0.9}

\title{Отчёт по лабораторной работе: \\ \Large{\textbf{"Дозы ионизирующих излучений"}}}
\author{Ужва Денис Романович, группа 15.Б08-фз}
\date{Дата проведения работы: 27.02.2018}

\topmargin = -2cm 
\hoffset = -1.5cm
\textheight = 700pt
\textwidth = 500pt

\newcolumntype{L}[1]{>{\raggedright\let\newline\\\arraybackslash\hspace{0pt}}m{#1}}
\newcolumntype{C}[1]{>{\centering\let\newline\\\arraybackslash\hspace{0pt}}m{#1}}
\newcolumntype{R}[1]{>{\raggedleft\let\newline\\\arraybackslash\hspace{0pt}}m{#1}}

\begin{document}
	\maketitle
	\smallskip
	
	\renewcommand{\abstractname}{Цель работы:}
	\begin{abstract}
		\noindent
		Изучить зависимость интенсивности излучения от расстояния и степень ослабления интенсивности излучения свинцовой пластиной. Расчитать мощность дозы в зоне действия пучка методом с фиксированным телесным углом. Измерить активность р/а препарата и мощность дозы методом $\gamma$-$\gamma$-совпадений. Сравнить методы определения мощности дозы.
	\end{abstract}
	
	\section{Теоретическое описание}

	\subsection{Зависимость интенсивности излучения от расстояния; степень ослабления интенсивности излучения свинцовой пластиной; расчёт мощности дозы в зоне действия пучка}
	Известно, что с расстоянием интенсивность излучения спадает по закону обратных квадратов (т.к. амплитуды электрического и магнитного полей пропорциональны -1 степени расстояния). Кратность снижения дозы же (ослабление интенсивности) защитным слоем зависит от толщины этого слоя $x$ и полного коэффициента остабления $\mu$ обратно экспоненциально:
	\begin{equation} \label{eq:1}
		I = I_0 \exp{(-\mu x)},
	\end{equation}
	где $I_0$ -- интенсивность излучения без защиты.

	Для расчёта мощности дозы следует сначала узнать активность источника. В части работы активность измеряется методом счёта с фиксированным телесным углом:
	\begin{equation} \label{eq:2}
		A = \frac{4\pi}{\Omega} \frac{N}{k},
	\end{equation}
	где $\Omega = \pi r^2 / R^2$ -- телесный угол, под которым излучение источника попадает в детектор (); $N$ -- скорость счёта; $k$ -- эффективность источника (для исследуемого образца ${}^{60}Co$ $k = 45\%$). Мощность рассчитывается следующим образом:
	\begin{equation} \label{eq:3}
		P = \frac{K_\gamma A}{R^2},
	\end{equation}
	где $K_\gamma = 3.0 \textnormal{ мЗв} \cdot \textnormal{см}^2 \cdot \textnormal{МБк}^{-1} \cdot \textnormal{ч}^{-1}$ -- ионизационная постоянная (для ${}^{60}Co$). $P$ в данном случае имеет размерность Зв/ч.

	\subsection{Измерение активности р/а препарата методом $\gamma$-$\gamma$-совпадений}
	Для данного метода требуется счётчик совпадений и два детектора. Его преимущество в том, что активность измеряется без каких либо дополнительных исследований характеристик детектора и источника:
	\begin{equation} \label{eq:4}
		A = N_0 = \frac{N_{\gamma1} N_{\gamma2}}{N_{coin}},
	\end{equation}
	где $N_0$ -- полное число распадов в секунду; $N_{\gamma1}$ и $N_{\gamma2}$ -- скорости счёта детекторов; $N_{coin}$ -- скорость счёта канала совпадений. Использование в измерениях схемы совпадений требует введения поправки на случайные совпадения:
	\begin{equation} \label{eq:5}
		N_{r} = 2 \tau N_{\gamma1} N_{\gamma2},
	\end{equation}
	где $\tau = 1$ мкс -- разрешающее время схемы совпадений.

	\subsection{Схемы установок}
	\begin{figure}[H]
  		\centering
  		\subfloat[$Установка 1$ с]{\label{fig:Sch1}\includegraphics[width=0.5\textwidth]{Sch1.jpg}}
  		\subfloat[$Установка 2$ с]{\label{fig:Sch2}\includegraphics[width=0.5\textwidth]{Sch2.jpg}}
  		\caption{\label{fig:Sch}Установки}
	\end{figure}
	Обозначения на схеме с первой установкой: 1 -- подставка; 2 -- сцинниляторный кристалл NaJ(Tl); 3 -- ФЭУ; 4 -- формирователь импульсов; 5 -- высоковольтный блок; 6 -- счётный прибор.
	\section{Ход работы}

	\subsection{Часть 1}
	Для данной части работы использовалась установка 1. Рабочее напряжение $U = 0.82$ кВ, экспозиция во всех измерениях 50 с. Первоначально был измерен фон до помещения препарата: 1484, 1468, 1449. После всех измерений с экземпляром также был определён фон: 1363, 1347, 1407. Таким образом, значение скорости счёта фона $N_{amb} = 1419.67\pm56.55$. Для определения расстояния от источника до приёмника необходимо было знать два соответствующих отсчёта. Для нахождения отсчёта приёмника были проанализированы полученные результаты следующим образом: зависимость значения величины $(N_{av} - N_{amb}) \cdot R^2 = N_{norm}$ от $R$ должна быть минимальной (для этого можно минимизировать разницу между максимумом и минимумом $ N_{norm}$). Таким образом, найденный отсчёт приёмника 7.3 см. В нижеследующей таблице $R$ -- разность отсчётов источника и приёмника:
		%\multicolumn{2}{| c |}{ среднее } & \multicolumn{3}{ c |}{ $(10.08\pm0.54)$ } & $(572\pm31)$ \\
	{\footnotesize
	\begin{longtable}{| l | l | l | l | l | l | l |}
		\hline
		$R$, см	& $N_1$	&	$N_2$	&	$N_3$	&	$N_{av}$	&	$N_{av} - N_{amb}$	&	$N_{norm}$					\\
		\hline
		17.7		& 4457		&	4342	&	4428	&	4409.00	&	$2989.33\pm116.36$	&	$936527.20\pm36454.42$		\\
		15.7		& 5630		&	5589	&	5593	&	5604.00	&	$4184.33\pm79.16$	&	$1031395.50\pm19512.15$	\\
		13.7		& 6789		&	6808	&	6687	&	6761.33	&	$5341.66\pm121.62$	&	$1002576.17\pm22826.86$	\\
		11.7		& 8495		&	8653	&	8473	&	8540.33	&	$7120.66\pm154.74$	&	$974747.15\pm21182.36$		\\
		9.7			& 11647	&	11974	&	11939	&	11853.33	&	$10433.66\pm236.09$	&	$981703.07\pm22213.71$		\\
		7.7			& 17522	&	17961	&	17726	&	17736.33	&	$16316.66\pm276.23$	&	$967414.77\pm16377.68$		\\
		5.7			& 31220	&	31347	&	31252	&	31273.00	&	$29853.33\pm122.6$	&	$969934.69\pm3983.27$		\\
		3.7			& 70798	&	70385	&	70104	&	70429.00	&	$69009.33\pm405.64$	&	$944737.73\pm5553.21$		\\
		\hline
	\caption{Результаты измерений числа сосчитанных импульсов} \label{tab:tab1}
	\end{longtable}}
	Соответствующие графики:
	\begin{figure}[H]
  		\centering
  		\subfloat[$(N_{av} - N_{amb})(R)$]{\label{fig:NNR}\includegraphics[width=0.5\textwidth]{NNR.jpg}}
  		\subfloat[$N_{norm}(R)$ с]{\label{fig:Nnorm}\includegraphics[width=0.5\textwidth]{Nnorm.jpg}}
  		\caption{\label{fig:Sch}Графики зависимостей искомых величин от $R$}
	\end{figure}
	Далее при фиксированном $R$ измерялась скорость счёта фона при отсутствии и наличии защитной свинцовой пластины. Толщина пластины $x$ измерялась несколько раз штангенциркулем: 3.0, 3.1, 2.9, 2.9 мм. Отсюда $x = 3.0\pm0.1$ мм. Измерение фона до помещения образца ${}^{60}Co$: 1363, 1347, 1407; после помещения образца: 1371, 1427, 1316. Отсюда $N_{amb} = 1371.83\pm40.23$. Расстояние между источником и приёмником взято $R = 14.7\pm0.1$ см. 
	{\footnotesize
	\begin{longtable}{| l | l | l | l | l |}
		\hline
		$N_1$	&	$N_2$	&	$N_3$	&	$N_{av}$	&	$N_{av} - N_{amb}$	\\
		\hline
		6115	&	6029	&	6139	&	6094.33	&	$4722.50\pm98.07$	\\
		5531	&	5593	&	5549	&	5557.67	&	$4185.84\pm72.13$	\\
		\hline
	\caption{Результаты измерений числа сосчитанных импульсов до и после помещения свинцовой пластины соответственно} \label{tab:tab2}
	\end{longtable}}
	Таким образом, отношение $N_{av} - N_{amb}$ для случаев "до" и "после" $N/N_0 = 0.866\pm0.027$. Для нахождения данного отношения из формулы \ref{eq:1} необходимо узнать $\mu$. Известно, что для ${}^{60}Co$ $E_{\gamma1} = 1173$ кэВ,  $E_{\gamma2} = 1332$ кэВ. Соответствующие $\mu_1 = 0.72$, $\mu_2 = 0.64$. Подставив данные значения в формулу \ref{eq:1}, получается $(I/I_0)_1 = 0.8996$ и $(I/I_0)_2 = 0.9102$. 

	Для получения значений $A$ и $P$ можно взять величины скорости счёта из таблицы \ref{tab:tab1}. Используя формулу \ref{eq:2}, получаются следующие значения $A$ (значение $N_{av} - N_{amb}$ в таблице -- скорость счёта):
	{\footnotesize
	\begin{longtable}{| l | l | l | l |}
		\hline
		$R$, см	&	$N_{av} - N_{amb}$			&	$A$, МБк	&	$A$, мкКю	\\	
		\hline
		17.7		&	$59.7866\pm2.3272$			&	0.0289		&	0.2111		\\
		15.7		&	$83.6866\pm1.5832$			&	0.0318		&	0.2323		\\
		13.7		&	$106.8332\pm2.4324$			&	0.0309		&	0.2257		\\
		11.7		&	$142.4132\pm3.0948$			&	0.0301		&	0.2199		\\
		9.7			&	$208.6732\pm4.7218$			&	0.0303		&	0.2213		\\
		7.7			&	$326.3332\pm5.5246$			&	0.0299		&	0.2184		\\
		5.7			&	$597.0666\pm2.4520$			&	0.0299		&	0.2184		\\
		3.7			&	$1380.1866\pm8.1128$		&	0.0292		&	0.2133		\\
		\hline
	\caption{Результаты вычисления $A$} \label{tab:tab3}
	\end{longtable}}
	Таким образом, усреднив полученные значения $A$, получается $A_{av} = 30.13\pm0.93 \textnormal{ кБк} = 220.06\pm6.82 \textnormal{ нКю}$. Тогда из формулы \ref{eq:3} ($R = 10$ см) $P = 0.90\pm0.03 \textnormal{ мкЗв/ч} = 102.72\pm3.17 \textnormal{ мкР/ч}$. Доза за время 6 ч $P = 5.42\pm0.17 \textnormal{ мкЗв} = 616.32\pm19.02 \textnormal{ мкР}$.

	Для сравнения были произведены измерения дозиметром:
	{\footnotesize
	\begin{longtable}{| l || l | l | l | l |}
		\hline
					&	$P_1$, мкЗв/ч	&	$P_2$, мкЗв/ч	&	$P_3$, мкЗв/ч	&	$P_{av}$, мкЗв/ч	\\	
		\hline
		Фон		&	0.13			&	0.19			&	0.17			&	$0.16\pm0.02$		\\
		\hline
		Образец	&	0.95			&	0.97			&	0.92			&	$0.95\pm0.03$		\\
		\hline
	\caption{Результаты вычисления $P$ дозиметром} \label{tab:tab4}
	\end{longtable}}
	Таким образом, $P = 0.79\pm0.05$ мкЗв/ч.
	
	\subsection{Часть 2}
	Измерения детекторами 1 и 2 проводились с экспозицией 100 с; каналом совпадений -- 1000 с. Полученные результаты (скорость счёта), где $N_r$ считается по формуле \ref{eq:5}:
	{\footnotesize
	\begin{longtable}{| l | l | l | l | l |}
					&	$N_1$					&	$N_2$					&	$N_{coin}$				&	$N_r$	\\	
		\hline
		Фон до		&	62.350					&	52.78					&	0.054					&	0.01	\\
		Фон после	&	65.580					&	55.21					&	0.010					&			\\
		Фон сред.	&	63.965					&	53.995					&	0.054					&	0.01	\\
		Образец	&	345.420				&	288.19					&	1.926					&	0.20	\\
		Обр. - фон	&	$281.455\pm20.865$	&	$234.195\pm18.694$	&	$1.872\pm1.620$		&	0.13	\\
		\hline
	\caption{Результаты измерений} \label{tab:tab4}
	\end{longtable}}
	$N_{coin}$ -- число совпадений, $N_r$ -- число случайных совпадений. Таким образом, активность образца по формуле \ref{eq:4} $A = 37.84\pm0.11$ кБк. Отсюда $P = 1.14\pm0.01$ мкЗв/ч.
	\section{Вывод}
	В ходе выполнения работы была проверена зависимость интенсивности ионизирующего излучения от расстояния между излучателем и приёмником: как и предсказывала теория, интенсивность убывает по закону обратных квадратов. Также было проверено, что интенсивность излучения экспоненциально убывает в поглощающем слое свинца. Были произведены измерения с целью определить активность и мощность дозы образца ${}^{60}Co$: по методу с фиксированным телесным углом $A_{av} = 30.13\pm0.93$ кБк,  $P = 0.90\pm0.03$ мкЗв/ч; по методу $\gamma$-$\gamma$-совпадений $A = 37.84\pm0.11$ кБк, $P = 1.14\pm0.01$ мкЗв/ч; с помощью дозиметра было установлено, что $P = 0.79\pm0.05$ мкЗв/ч. Таким образом, видно, что оба метода вполне пригодны для измерения активности и мощности дозы.
	\end{document}