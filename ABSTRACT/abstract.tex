\begin{abstract}

В рамках квантовой хромодинамики большую роль играют феноменологические подходы к рассмотрению процессов рассеяния одних ядер на других, для проверки которых прибегают к помощи компьютерных симуляций на дискретной решётке, моделирующей сечение сталкивающихся ядер.
При всём многообразии реализаций и успешности имплементации решёточных симуляций интереса заслуживает рассмотрение непрерывной модели, более детально отображающей действительность, что и было сделано в данной работе. 
В качестве взаимодействующих ядер были взяты ядра свинца, радиус которых $R_{Pb} \approx 7.5$ фм, и основной задачей являлась проверка корреляций множественности ($nn$) и поперечного импульса ($p_tp_t$) при учёте и без учёта слияния цветных кварк-глюонных струн -- растянутых из-за большого импульса налетающих частиц полей сильного взаимодействия -- до начала процесса адронизации. 
В результате вычислений оказалось, что расчёты на непрерывной области взаимодействия совпадают с расчётами на дискретной решётке и, вместе с тем, с теорией, дающей асимптотическую оценку коэффициентов корреляции в зависимости от плотности струн при больших её значениях.
На данном этапе работы особых преимуществ непрерывных симуляций по сравнению с дискретными не наблюдалось, однако, благодаря проверке на непрерывной области можно заключить, что более грубый метод вычислений на решётке действительно работает правильно, и им можно пользоваться для упрощения вычислений.

\end{abstract}
