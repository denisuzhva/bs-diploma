\section{Введение}
\subsection{Модельный статистический анализ в ФВЭиЭЧ}
В современной физике высоких энергий и элементарных частиц большую роль играют компьютерные симуляции, позволяющие набирать выборку статистических данных для анализа и, впоследствии, верифицировать феноменологические гипотезы. 
Симуляции представляют из себя, во-первых, модельное воспроизведение так называемых ``событий'' (event generation) -- физических процессов, необходимых для наблюдения определённых явлений, в виде сгенерированного на основе гипотезы набора величин; во-вторых, машинные расчёты по генерируемым значениям величин -- для сравнения результатов симуляций с аналогичными результатами на основе экспериментальных данных. 
К типичным моделируемым в рамках физики высоких энергий процессам можно отнести, к примеру, процесс множественного рождения частиц при столкновении (рассеянии) ядер в экспериментах на коллайдерах.
%%%
\subsection{Модель кварк-глюонных струн в КХД}
Квантовая хромодинамика (КХД) возникла в резульате развития кварковой модели \cite{YndurainQCD}, и на данный момент является общепризнанной теорией сильных взаимодействий. 
Идея КХД заключается в присвоении кваркам так называемого цветового заряда, отвечающего за сильное взаимодействие подобно электрическому -- за электромагнитное. 
Всего цветовых зарядов три (плюс три антицвета): красный, зелёный и синий (и парные антикрасный, антизелёный и антисиний соответственно). 
Характерная особенность сильного взаимодействия заключается в конфайнменте -- невозможности существования кварков по отдельности: данные частицы могут быть лишь в ``обесцвеченной'' (``белой'') совокупности (обесцвечивание происходит по закону аддитивного смешения цветов, например, красного с зелёным и синими или зелёного с антизелёным). 
Причиной конфайнмента служит возрастание энергии поля сильного взаимодействия при отдалении кварков друг от друга, в отличие от, например, электромагнитного поля. 
Стоит отметить также, что ``цвета'' кварков -- лишь удобная репрезентация группы $SU(3)$, классифицирующей адроны, и не имеют отношения к оптическим цветам.

Столкновения ядер в коллайдере можно разделить на две категории: мягкие и жёсткие. 
При жёстком рассеянии партонов (кварков и гюонов) встречных ядер  сталкиваются ``лоб в лоб'', порождая так называемые адронные струи с большим поперечным импульсом $p_t$. 
Первоначально считалось, что при сверхвысоких энергиях БАК
жёсткие процессы окажутся доминирующими, а мягкое рассеяние можно воспринимать как возмущение. 
Важно также отметить, что расчёты в рамках жёсткой области можно проводить из первичных принципов, используя теорию возмущений КХД. 
Позже было выяснено, что рассеяние в мягкой области нельзя свести к возмущению, так как на больших масштабах нарушаются условия применимости теории возмущений КХД \cite{SoftQCD1,SoftQCD2}.

С точки зрения феноменологии удобной описательной моделью мягкой адронизации является модель цветных кварк-глюонных струн\cite{ColorStringsModel1,ColorStringsModel2}. 
Суть модели заключается в понимании кварк-глюонного поля в партонном облаке как совокупности струн, связывающих эти партоны. 
В момент столкновения двух ядер струны из встречных партонных облаков переплетаются, натягиваются и рвутся (фрагментируют); ``обрывки'' струн при этом вследствие конфайнмента рекомбинируют в адроны. 
%%%
\subsection{Слияние струн}
При ядро-ядерных столкновениях на энергиях порядка LHC плотность струн может оказаться столь высокой, что приходится учитывать нелинейные явления, вызванные эффектом взаимодействия струн между собой, которые влияют на множественность \cite{StringFusion}.

Для случая рассеяния тяжелых ядер М.А. Брауном и К. Пахаресом была предложена модификация струнной модели, учитывающая процессы возможного слияния первичных струн до начала процесса их фрагментации \cite{PreFusion1, PreFusion2}. 
Слившиеся струны образуют так называемые кластеры (совокупности слившихся струн, отдельные от других подобных совокупностей), а множественность адронов зависит от параметров того или иного кластера, причём чем больше плотность струн (больше энергия), тем слабее прирост плотности струн влияет на прирост значения средней множественности \cite{MulReduction}: адронизация происходит от одной большой ``струны'' вместо суммы составляющих её первичных струн. 
Стоит отметить, что учёт нелинейности можно определять как при локальном сложении полей (локальное слияние), так и при сложении по всему поперечному сечению каждого кластера (глобальное слияние) \cite{MulReduction}.

За счёт слияния струн будут меняться корреляции множественности ($nn$) и поперечного импульса ($p_t p_t$) адронов в переднем и заднем быстротных окнах, возникающие благодаря флуктуациям числа участников столкновения (в струнной модели -- числа струн). 
Причём на корреляции $p_t p_t$ учёт кластеризации струн оказывает большее влияние, нежели на $nn$ \cite{MulReduction, PtPtCorr}. 
Связано это с тем, что усреднённая по событию величина $\langle p_t \rangle$ является интенсивной, в отличие от экстенсивной $\langle n \rangle$, и поэтому практически не зависит от числа струн, принимая малые значения при отбрасывании эффекта слияния. 
Однако, если кластеризацию учесть, начнут проявляться ощутимые отклонения $\langle p_t \rangle$ от её значения для одиночной струны, влекущие за собой наличие $p_t p_t$ корреляций.
%%%
\subsection{Генераторы событий Монте-Карло в мягкой области}
Важной частью анализа столкновений ядер являеются ``Монте-Карловские'' решёточные симуляции \cite{TransLattice1}: моделируется поперечное сечение ядро-ядерного столкновения, в котором происходит наложение распределённых сечений участвующих в адронизации струн. 
По измерениям площадей этих сечений и плотности струн можно рассчитывать $nn$, $p_t p_t$ и $p_t n$ корреляции в различных быстротных окнах.
%%%
\subsection{Постановка задачи}
При всём удобстве  и высокой скорости расчётов решёточных симуляций интересно также рассмотреть более детальный подход: вместо дискретной решётки использовать непрерывную область сечения, так как непрерывная область позволяет точнее смоделировать сечение, что влечёт за собой более качественную проверку теории. 
Именно такой подход был использован для определения и исследования $nn$ и $p_tp_t$ корреляций в данной работе. 

Результатом данной работы является качественное и количественное сравнение $nn$ и $p_tp_t$ корреляций в переднем и заднем (FB -- Forward-Backward) быстротных окнах с учётом слияния струн и без, определение влияния учёта эффекта слияния струн на множественность адронов, а также сравнение результатов с аналитическими асимптотическими формулами, полученными в работах \cite{bStatement} и \cite{dissert}.
