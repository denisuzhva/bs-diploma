\section{Приложение}
\subsection{Изображения}
%\imgh{15cm}{QCD}{Некоторые ``бесцветные'' комбинации кварков, способные существовать}{QCD}
\imgh{14cm}{DevSch}{Принципиальная схема работы генератора}{DevSch}
\imgh{14cm}{b_eta/b_nn}{График зависимости коэффициента корреляции $b_{nn}$ от $\langle \eta \rangle$ (С++, усреднено по 40000 симуляциям)}{bnn}
\imgh{14cm}{b_eta/b_pp}{График зависимости коэффициента корреляции $b_{p_tp_t}$ от $\langle \eta \rangle$ (С++, усреднено по 40000 симуляциям)}{bpp}
%%%
\subsection{Таблица со значениями корреляционного коэффициента}
{\scriptsize
\begin{longtable}{| l | l | l || l | l | l |}
	\hline
	$\langle \eta \rangle$	&	$b_{nn}$ (fusion)	&	$b_{nn}$ (no fusion)	&	$\langle \eta \rangle$	&	$b_{nn}$ (fusion)	&	$b_{nn}$ (no fusion)	\\
	\hline
	0.090	&	0.468	&	0.493	&	2.970	&	0.173	&	0.498	\\
	0.225	&	0.448	&	0.495	&	5.040	&	0.115	&	0.507	\\
	0.495	&	0.421	&	0.505	&	10.980	&	0.079	&	0.502	\\
	0.990	&	0.368	&	0.505	&	-		&	-		&		-	\\
	\hline
	\caption{Значения $\langle \eta \rangle$ и соответствующие $b_{nn}$ (fusion) и $b_{nn}$ (no fusion) к рис. \ref{fig:bnn}} \label{tab:bnn}
\end{longtable}}

{\scriptsize
\begin{longtable}{| l | l | l || l | l | l |}
	\hline
	$\langle \eta \rangle$	&	$b_{p_tp_t}$ (fusion)	&	$b_{p_tp_t}$ (no fusion)	&	$\langle \eta \rangle$	&	$b_{p_tp_t}$ (fusion)	&	$b_{p_tp_t}$ (no fusion)	\\
	\hline
	0.090	&	-0.008	&	-0.007	&	2.970	&	0.055	&	-0.003	\\
	0.225	&	0.004	&	-0.002	&	5.040	&	0.045	&	-0.007	\\
	0.495	&	0.010	&	-0.009	&	10.980	&	0.033	&	0.009	\\
	0.990	&	0.023	&	-0.002	&	-		&	-		&		-	\\
	\hline
	\caption{Значения $\langle \eta \rangle$ и соответствующие $b_{p_tp_t}$ (fusion) и $b_{p_tp_t}$ (no fusion) к рис. \ref{fig:bpp}} \label{tab:bnn}
\end{longtable}}

