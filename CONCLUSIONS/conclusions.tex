\section{Заключение}
Феноменологическая модель кварк-глюонных струн неоднократно доказывала свою превосходную адекватность описания хромодинамических процессов, происходящих в мягкой области рассеяния ультрарелятивистских ядер. Её важность для более глубокого понимания принципов Стандартной модели неоспорим, так как она хорошо описывает взаимодействие одного кварк-глюонного поля с другими. 

В данной работе приведены результаты модельного теоретического анализа рассеяний ультрарелятивистских ядер при высоких энергиях порядка RHIC (200 ГэВ), LHC (13 ТэВ). Для получения этих результатов была постепенно, в 4 этапа, разработана специальная программа -- генератор событий -- моделирующая поперечное сечение столкновения ядер в виде непрерывно распределённых сечений струн фиксированного радиуса, а также проводящая расчёты множественности рождённых адронов и их корреляций, выражаемых корреляционных коэффициентом $b$, в переднем и заднем быстротных окнах с учётом слияния (пересечения) данных струн и без. Результатами являются построенные графики зависимости $\langle n_{f} \rangle / \langle n_{nf} \rangle$ от $\langle \eta \rangle$ -- для проверки ослабления зависимости множественности в одном событии от изменения числа струн в сечении при большом количестве этих струн и зависимости корреляционных коэффициентов $b_{nn}$ и $b_{p_tp_t}$ от $\langle \eta \rangle$ -- для проверки влияния учёта кластеризации струн на FB-корреляции множественности; величина $\langle \eta \rangle$ играет роль средней плотности струн в области взаимодействия. Из вида этих графиков можно сделать два вывода: 
\begin{enumerate}[label=\arabic*.]
\item	При учёте слияния струн усреднённая по событиям при фиксированном $\langle \eta \rangle$ множественность адронов действительно получает более слабый прирост при увеличении плотности струн на больших значениях $\langle \eta \rangle$ (таких, что почти вся область взаимодействия покрыта струнами, слившимися в один кластер).
\item	Корреляционные коэффициенты $b_{nn}$ и $b_{p_tp_t}$, как и ожидалось, сильно зависят от введения учёта слияния струн: количественное различие как среднее процентное соотношение составляет $\langle D \rangle _{nn} = 40.7 \%$, $\langle D \rangle _{p_tp_t} = 97.7 \%$. 
\end{enumerate}
Таким образом, основные поставленные задачи были выполнены.

Разработка генератора на текущий момент ещё не закончена. В перспективе планируется использование более реалистичных распределений струн (например, потенциал Вудса-Саксона), введение переменной центральности столкновений, а также оптимизация кода.