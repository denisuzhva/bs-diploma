\section{Заключение}
Феноменологическая модель кварк-глюонных струн неоднократно доказывала свою адекватность при описании хромодинамических процессов, происходящих в мягкой области рассеяния ультрарелятивистских ядер. Эта модель позволяет более глубоко понять специфику сильного взаимодействия при высоких энергиях через образование кластеров цветных кварк-глюонных струн. 

В данной работе приведены результаты модельного теоретического анализа рассеяний ультрарелятивистских ядер при высоких энергиях порядка RHIC (200 ГэВ), LHC (13 ТэВ). 
Для получения этих результатов была разработана специальная программа -- генератор событий -- моделирующая поперечное сечение столкновения ядер в виде непрерывно распределённых сечений струн фиксированного радиуса, а также проводящая расчёты множественности рождённых адронов и их корреляций, выражаемых корреляционных коэффициентом $b$, в переднем и заднем быстротных окнах с учётом слияния (пересечения) данных струн и без. 
Результатами являются построенные графики зависимости корреляционных коэффициентов $b_{nn}$ и $b_{p_tp_t}$ от $\langle \eta \rangle$ -- для проверки влияния учёта кластеризации струн на FB-корреляции множественности; величина $\langle \eta \rangle$ играет роль средней плотности струн в области взаимодействия. 
Из вида этих графиков можно сделать два вывода: 
\begin{enumerate}[label=\arabic*.]
\item	Корреляционные коэффициенты $b_{nn}$ и $b_{p_tp_t}$, как и ожидалось, сильно зависят от введения учёта слияния струн: количественное различие как среднее процентное соотношение составляет $\langle D \rangle _{nn} = 40.7 \%$, $\langle D \rangle _{p_tp_t} = 58.7 \%$; причём на корреляции поперечного импульса учёт слияния оказывает больший эффект, чем на $b_{nn}$.
\item   Полученные в настоящей работе результаты расчетов коэффициентов $nn$ и $p_tp_t$ корреляций, выполненных с учетом в каждом событии реальной геометрии образующихся струнных кластеров, хорошо согласуются при большой плотности струн с асимптотическими оценками, полученными в приближенном варианте модели со слиянием струн на поперечной решетке.
\end{enumerate}
Таким образом, основные поставленные задачи были выполнены. Из вышеперечисленного можно заключить, что поскольку результаты расчётов, выполненные с использованием генератора событий, учитывающего реальную геометрию образующихся струнных кластеров, совпадают с оценками, полученными в приближенном варианте модели с дискретной решеткой, то это дискретное приближение может использоваться для упрощения вычислений.

Разработка генератора на текущий момент ещё не закончена. В перспективе планируется использование более реалистичных распределений струн (например, потенциал Вудса-Саксона), введение переменной центральности столкновений, а также оптимизация кода.
