\begin{thebibliography}{24}
\bibitem{EventGeneration}
D. Perret-Gallix (2013).
\textit{Computational Particle Physics for Event Generators and Data Analysis}.
J. Phys. Conf. Ser.
\textbf{454}.
012051
\href{https://arxiv.org/abs/1301.1211}{arXiv:1301.1211}

\bibitem{HiggsSt}
L. Taylor (2012).
\textit{Observation of a New Particle with a Mass of 125 GeV.}
CMS Experiment, CERN
\url{http://cms.web.cern.ch/news/observation-new-particle-mass-125-gev}

\bibitem{YndurainQCD}
F.J. Yndurain (1993).
\textit{Quantum Chromodynamics: An Introduction to the Theory of Quarks and Gluons}.
ISBN 978-3-540-33210-7

\bibitem{HadronJet}
S. Sapeta (2016).
\textit{QCD and Jets at Hadron Colliders.}
Prog. Part. Nucl. Phys.
\textbf{89}.
1-55
\href{https://arxiv.org/abs/1511.09336v2}{arXiv:1511.09336}

\bibitem{SoftQCD1}
E. Nurse (2011).
\textit{Soft-QCD at Hadron Colliders} (slides).
\url{http://www.hep.ucl.ac.uk/~mw/Post_Grads/2011-12/SoftQCDLecture.pdf}

\bibitem{SoftQCD2}
D.S. Cerci (2017).
\textit{Soft and Hard QCD Processes in CMS} (slides).
\url{https://indico.cern.ch/event/614845/contributions/2728799/attachments/1529660/2398415/13_DSunarCerci.pdf}

\bibitem{ColorStringsModel1}
A.A. Kaidalov (1982).
\textit{The quark-gluon structure of the pomeron and the rise of inclusive spectra at high energies}.
Phys. Lett. B
\textbf{116}.
459-463
\url{https://doi.org/10.1016/0370-2693(82)90168-X}

\bibitem{ColorStringsModel2}
A.A. Kaidalov, K.A. Ter-Martirosyan (1982).
\textit{Pomeron as quark-gluon strings and multiple hadron production at SPS-Collider energies}.
Phys. Lett. B
\textbf{117B}.
247-251
\url{https://doi.org/10.1016/0370-2693(82)90556-1}

\bibitem{ColorStringsModel3}
A. Capella, U. Sukhatme, Tan Chung-I, J. Tran Thanh Van (1979).
\textit{Jets in small-$p_T$ hadronic collisions, universality of quark fragmentation, and rising rapidity plateaus}.
Phys. Lett. B
\textbf{81}.
68-74
\url{https://doi.org/10.1016/0370-2693(79)90718-4}

\bibitem{ColorStringsModel4}
A. Capella, U. Sukhatme, Tan Chung-I, J. Tran Thanh Van (1994).
\textit{Dual parton model}.
Phys. Rep.
\textbf{236}.
225-329
\url{https://doi.org/10.1016/0370-1573(94)90064-7}

\bibitem{SeaQarkStrings}
G.I. Lykasov, M.N. Sergeenko (1996).
\textit{Semihard Hadron Processes and Quark-gluon String Model}.
Z.Phys.
\textbf{C70}.
455-462
\href{https://arxiv.org/abs/hep-ph/9502316v1}{arXiv:hep-ph/9502316}

\bibitem{StringFusion}
T.S. Biro, H.B. Nielsen, J. Knoll (1984).
\textit{Colour rope model for extreme relativistic heavy ion collisions}.
Nucl. Phys. B 
\textbf{245}.
449-468
\url{https://doi.org/10.1016/0550-3213(84)90441-3}

\bibitem{PreFusion1}
M.A. Braun and C. Pajares (1992).
\textit{Particle production in nuclear collisions and string interactions}.
Phys. Lett. B 
\textbf{f287}.
154

\bibitem{PreFusion2}
M.A. Braun and C. Pajares (1993).
\textit{A probabilistic model of interacting strings}.
Nucl. Phys. B
\textbf{f390}.
542 

\bibitem{PreFusion3}
N.S. Amelin, N. Armesto, M.A. Braun, E.G. Ferreiro and C.Pajares (1994).
\textit{Long and Short Range Correlations: A Signature of String Fusion}.
Phys. Rev. Lett.
\textbf{f73}
2813

\bibitem{PreFusion4}
V.V. Vechernin, C. Pajares, M.A. Braun (2001).
\textit{Forward-backward multiplicity correlations, low $p_T$ distributions in the central region and the fusion of colour strings}.
CERN, Geneva.
ALICE Collaboration.
\url{http://cds.cern.ch/record/689450}

\bibitem{MulReduction}
В.В. Вечернин, Р.С. Колеватов (2007).
\textit{О корреляциях множественности и $p_t$ в столкновениях ультрарелятивистских ионов}.
Ядерная физика
\textbf{70}
1846-1857
\url{http://elibrary.ru/item.asp?id=9549732}

\bibitem{PtPtCorr}
В.В. Вечернин, Р.С. Колеватов (2007).
\textit{Дальние корреляции между поперечными импульсами заряженных частиц в релятивистских ядерных столкновениях}
Ядерная физика
\textbf{70}
1858-1867
\url{http://elibrary.ru/item.asp?id=9549733}

\bibitem{MonteCarlo1}
S. Weinzierl (2000).
\textit{Introduction to Monte Carlo methods}.
Topical lectures given at the Research School Subatomic Physics, Amsterdam, June 2000
\href{https://arxiv.org/abs/hep-ph/0006269v1}{arXiv:hep-ph/0006269}

\bibitem{MonteCarlo2}
N. Srimanobhas (2010).
\textit{Introduction to Monte Carlo for Particle Physics Study} (slides).
\url{https://indico.cern.ch/event/92209/contributions/2114409/attachments/1098701/1567290/CST2010-MC.pdf}

\bibitem{TransLattice1}
V.V. Vechernin, R.S. Kolevatov (2004).
\textit{Cellular Approach to Long-Range $p_t$ and Multiplicity Correlations in the String Fusion Model}.
Vestnik SPbU
\textbf{ser.4, no.4}.
11-27
\href{https://arxiv.org/abs/hep-ph/0305136}{arXiv:hep-ph/0305136}

\bibitem{TransLattice2}
V.V. Vechernin, R.S. Kolevatov (2004).
\textit{Simple Cellular Model of Long-Range Multiplicity and $p_t$ Correlations in High-Energy Nuclear Collisions}.
Vestnik SPbU
\textbf{ser.4, no.2}.
12-23
\href{https://arxiv.org/abs/hep-ph/0305136}{arXiv:hep-ph/0305136}

\bibitem{bStatement}
M.A. Braun, C. Pajares (2000)
\textit{Inplication of percolation of colour strings on multiplicities, correlations and the transverse momentum}.
Eur. Phys. J.
\textbf{C16}.
349–359
\href{https://arxiv.org/abs/hep-ph/9907332v1}{arXiv:hep-ph/9907332}

\bibitem{RHICandLHC}
J. Dias de Deus, A.S. Hirsch, C. Pajares, et al. (2012).
\textit{Clustering of color sources and the shear viscosity of the QGP in heavy ion collisions at RHIC and LHC energies}.
Eur. Phys. J. C 72:2123
\url{https://doi.org/10.1140/epjc/s10052-012-2123-x}

\bibitem{Mu0}
D. Kharzeev, M. Nardi (2001).
\textit{Hadron production in nuclear collisions at RHIC and high density QCD}.
Phys. Lett. B
\textbf{507}
121-128
\href{https://arxiv.org/abs/nucl-th/0012025}{arXiv:nucl-th/0012025}

\end{thebibliography}