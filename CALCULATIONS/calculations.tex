\section{Результаты расчётов}
%%
%%
\subsection{Вычисление корреляционного коэффициента}
Первые 3 этапа разработки пройдены успешно.
Теперь, когда появилась возможность генерировать множественности (а значит, и поперечные импульсы), можно приступить к 4 этапу -- расчётам коэффициентов корреляции $b_{nn}$ и $b_{p_tp_t}$.
Важно отметить, что количество струн здесь разыгрывалось по функции вероятности Пуассона (см. формулу \ref{eq:poisson}), поэтому $\eta$ и $\langle \eta \rangle$, вообще говоря, различны.

Значения множественности $n_F$ и $n_B$ (и $p_F$ и $p_B$) для каждого события подставляются в формулы \ref{eq:bCalc}, затем строятся графики зависимости $b(\langle \eta \rangle)$. Полученные графики представлены на рис. \ref{fig:bnn} и \ref{fig:bpp} в приложении; значения $\langle \eta \rangle \approx 3$ соответствует энергиям RHIC, а $\langle \eta \rangle \approx 11$ -- LHC. 
Как можно заметить, учёт кластеризации достаточно сильно влияет на поведение $b$: количественно различие $b_{\textnormal{fusion}}$ и $b_{\textnormal{no fusion}}$ можно выразить средним процентным соотношением $\langle D \rangle$ (как для $b_{nn}$, так и для $b_{p_tp_t}$):
\begin{equation} \label{eq:bfbn}
	\langle D \rangle = \bigg \langle 1 - \frac{min(b_{\textnormal{fusion}}, b_{\textnormal{no fusion}})}{max(b_{\textnormal{fusion}}, b_{\textnormal{no fusion}})} \bigg \rangle,
\end{equation}
усреднение ведётся по всем $\langle \eta \rangle$; тогда для проведённых симуляций $\langle D \rangle _{nn} = 40.7 \%$, а $\langle D \rangle _{p_tp_t} = 58.7 \%$. Нетрудно заметить, что, как и было предсказано, учёт слияния струн сильнее влияет на коэффициент корреляции поперечных импульсов, нежели множественностей.

Вычисления $b_{nn}(\langle \eta \rangle)$ и $b_{p_tp_t}(\langle \eta \rangle)$ в случае со слиянием струн занимали приблизительно 3 часа для 20000 симуляций, а в случае без слияния -- около 3 минут, тоже для 20000. Столь большая разница в скоростях расчёта связана с отсутствием необходимости во втором случае вводить граф пересечений струн и пресчитывать кластеры.
