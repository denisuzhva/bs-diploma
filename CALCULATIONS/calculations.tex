\section{Результаты расчётов}
%%
\subsection{Перколяционный переход}
\subsubsection{Равномерное распределение струн}
Как указано выше, первоначально алгоритм расчётов был реализован в среде MATLAB. Функции, которые были применены в разработке на этом языке, аналогичны описанным выше для C++ за исключением уже встроенной в MATLAB функции \texttt{conncomp(G)}, возвращающей компоненты связности графа \texttt{G}. 

С помощью MATLAB первые два этапа работы были преодолены с положительным результатом.
Данный результат заключается в наблюдении так называемого ``перколяционного перехода'': фазового перехода кластеризации, при котором в некоторой области $[\eta, \eta + \Delta \eta]$ резко возрастает размер максимального кластера при малом увеличении $\eta$, где под ``размером'' максимального кластера подразумевается как количество струн в нём $N_{cl}$, так и его площадь $S_{cl}$ -- расчёт проведён для обеих величин. 
Для наблюдения перколяционного перехода достаточно положить $\eta = \langle \eta \rangle$ и $N = \langle N \rangle$, поэтому для данного расчёта угловые скобки опущены. Также для простоты кода радиус струны и радиус ядра были пропорционально уменьшены до 1 и 0.03 условных единиц соответственно. 

С целью наблюдения вышеописанного явления удобно построить нормированные графики зависимостей $N_{cl}/N$ и $S_{cl}/S$ от $\eta$, что и было сделано (рис. \ref{fig:NclNetaMATLAB} и \ref{fig:SclSetaMATLAB} в приложении) для небольшого количества симуляций. 
На обоих графиках можно выделить три интервала: при $\eta$ от 0 до $0.2 \div 0.3$; $\eta$ от $0.2 \div 0.3$ до $1.8 \div 1.9$; $\eta$ от $1.8 \div 1.9$ до $10.2$. 
Первый интервал соответствует ситуации малой плотности струн в области взаимодействия, из-за чего они редко перекрываются, образуя кластеры.
В данной области значение $N_{cl}$ близко к 1, в то времся как $N$ растёт, что приводит к образованию небольшой ``ямки'' на графике в данной области значений $\eta$.
Далее по мере роста $N$ и приближения ко второму интервалу происходит уплотнение струн в области взаимодействия и воссоединение небольших кластеров, причём каждая новая струна может соединить один или несколько максимальных кластеров в один значительно больший кластер (например, три близких кластера из 20 струн каждый могут соединиться добавленной струной в кластер из 61 струны); и такая соединительная струна может попасться в любой симуляции при заданном $N$, что вызывает довольно большую погрешность значений $N_{cl}/N$ и $S_{cl}/S$ во втором интервале значений $\eta$. 
В третьем интервале происходит своеобразное насыщение: при больших $N$ струны всё плотнее заполняют область взаимодействия, сливаясь в единственный большой кластер; как следствие, $N_{cl} \rightarrow N$ и $S_{cl} \rightarrow S$.

\subsubsection{Параболическое распределение}
Несмотря на высокую скорость разработки подобных математических программ в MATLAB, их производительность получается очень низкой ввиду высокого уровня абстракции языка, влекущей сильную перегруженность машинного кода. Так, на расчёт 60-ти симуляций с вычислением $S_{cl}/S$ для множества $\eta$ ушло 1-2 часа. Так как далее расчёты будут только сложнее, а симуляций потребуется почти в тысячи раз больше, было принято решение переписать алгоритмы на C++ и работать дальше только с ним.

Для ещё большей уверенности в корректности работы программы были проведены расчёты зависимости $N_{cl}/N$ и $S_{cl}/S$ от $\eta$ при разбрасывании струн в область взаимодействия не равномерно, а с помощью параболического распределенияi (формула \ref{eq:parab}). 
Соответствующие графики приведены на рис. \ref{fig:NclNeta} и \ref{fig:SclSeta} в приложении. 
Можно заметить, что область перколяционного перехода теперь не такая резкая, какой была при вычислениях с равномерным распределением. 
Данный результат связан с тем, что, так как струны разыгрываются теперь плотнее к центру, единый большой кластер начинает образовываться при меньших значениях $\eta$, но при всём вышесказанном это уплотнение препятствует быстрому насыщению, группируя много струн в центре и оставляя периферию области взаимодействия для образования небольших кластеров, тогда как при равномерном распределении все струны сцепляются и нигде не накапливаются.
Погрешности $\Delta \eta$ (см. таблицу \ref{tab:NclNeta} в приложении) связаны с непостоянным значенем $\eta$ для каждого события из-за разброса значений площади взаимодействия $S$, вычисляемой для расчёта с параболическим распределением не как площадь единичного круга, а как суммарная площадь всех кластеров, так как даже при больших $\eta$ площадь взаимодействия может быть меньше площади единичного круга из-за уплотнения струн.

С помощью программы было проведёно 3638 симуляций с вычислением как $N_{cl}/N$, так и $S_{cl}/S$, и на оба расчёта ушло 5 часов суммарно. Это значит, что расчёт 60 симуляций занял бы приблизительно 5 минут, что намного быстрее вычислений в MATLAB. 
%%
\subsection{Учёт слияния струн до адронизации}
Описанные выше результаты получены благодаря преодолению первых двух этапов разработки. 
Чтобы пронаблюдать явно учёт влияния кластеризации на множественность адронов в событии и проверить слова из \cite{MulReduction}, необходимо работать уже с третьим этапом. 
Всюду далее в вычислениях используется равномерное распределение струн по области взаимодействия.

Описанный выше процесс генерации множественности в программе используется для сравнения множественности адронов $\langle n_{f} \rangle$ (f -- fusion), рождённых каждым кластером в совокупности событий, с множественностью $\langle n_{nf} \rangle$ (nf -- no fusion), полученную от каждой струны по отдельности (усреднение данных величин ведётся по совокупности событий при фиксированной величине $\eta$). 
Сравнение можно провести при помощи графика зависимости $\langle n_{f} \rangle / \langle n_{nf} \rangle$ от $\eta$ (рис. \ref{fig:nn0eta}). 
На данном графике отчётливо наблюдается монотонное убывание отношения $\langle n_{f} \rangle / \langle n_{nf} \rangle$ при значениях $\eta$ характерных для перколяционного перехода и выше, что свидетельствует об ослаблении влияния увеличения количества струн на прирост множественности при учёте их слияния. При этом в области низких плотностей струн данное отношение близко к единице, что вполне естественно, так как струны перекрываются редко.
%%
\subsection{Вычисление корреляционного коэффициента}
Вычисления на данном этапе (четвёртом) проводятся с использованием параболоидного распределения (формула \ref{eq:parab}), а область взаимодействия фиксируется как единичная окружность. 
Значения множественности $n_F$ и $n_B$ для каждого события подставляются в формулу \ref{eq:nfnb}, затем строится график зависимости $b(\eta)$. 
Полученные графики представлены на рис. \ref{fig:bEta} и \ref{fig:bEtaScaled} (отмасштабирован по оси $b$ для наглядности) в приложении; значения $\eta \approx 2.80$ соответствует энергиям RHIC, а $\eta \approx 10.35$ -- LHC. 
Нетрудно заметить, что учёт кластеризации слабо влияет на поведение $b$: количественно различие $b_{\textnormal{fusion}}$ и $b_{\textnormal{no fusion}}$ можно выразить средним процентным соотношением $\langle D \rangle$ 
\begin{equation} \label{eq:bfbn}
	\langle D \rangle = \bigg \langle 1 - \frac{min(b_{\textnormal{fusion}}, b_{\textnormal{no fusion}})}{max(b_{\textnormal{fusion}}, b_{\textnormal{no fusion}})} \bigg \rangle,
\end{equation}
усреднение ведётся по всем $\eta$; тогда для проведённых симуляций $\langle D \rangle = 1.57 \%$. Интересно отметить, что при слиянии струн изгиб графика $b(\eta)$ становится менее пологим, нежели у графика $b(\eta)$ без слияния.

В качестве проверки правильности вычисления $b$ программой можно рассмотреть график зависимости $b(\eta)$ при симуляциях с $N = \langle N \rangle$, т.е. вместо равномерно распределённого значения $N$ взять среднее. Ожидается, что в таком случае коррелятор $b$ будет близок к нулю, так как при расчётах с фиксированным $N$ корреляции наблюдать невозможно, и числитель в формуле \ref{eq:nfnb} становится малым. Действительно, первое слагаемое в числителе можно расписать как
\begin{align*}
	\langle n_F n_B \rangle = \langle n_F \rangle \langle n_B \rangle + K_{n_F n_B}, \quad K_{n_F n_B} = \langle (n_F - \langle n_F \rangle)(n_B - \langle n_B \rangle) \rangle,
\end{align*}
где $K_{n_F n_B}$ -- корреляционный момент случайных величин $n_F$ и $n_B$, тогда второе слагаемое в числителе сократится с первым из выражения $\langle n_F n_B \rangle$, и останется лишь $K_{n_F n_B}$. При постоянном $N$ в каждой симуляции величины $n_F$ и $n_B$ становятся независимыми, так как генерируются последовательно для одной и той же конфигурации кластеров технически независимо, следовательно
\begin{align*}
	K_{n_F n_B} = \langle (n_F - \langle n_F \rangle)(n_B - \langle n_B \rangle) \rangle = \langle n_F - \langle n_F \rangle \rangle \langle n_B - \langle n_B \rangle \rangle = 0.
\end{align*}
Из графика зависимости $b(\eta)$ (симуляции проведены с учётом слияния струн) в приложении (рис. \ref{fig:bEtaConstN} и \ref{fig:bEtaConstNScaled}) видно, что на практике вышеуказанное предположение выполняется.

Вычисления $b(\eta)$ в случае со слиянием струн занимали приблизительно 2 часа для 10000 симуляций, а в случае без слияния -- около 2 минут, тоже для 10000. Столь большая разница в скоростях расчёта связана с отсутствием необходимости во втором случае вводить граф пересечений струн и пресчитывать кластеры.
